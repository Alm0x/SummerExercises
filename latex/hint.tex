\documentclass[12pt]{article}

\usepackage[T2A]{fontenc}
\usepackage[utf8]{inputenc}
\usepackage[english,russian]{babel}
\usepackage{amsmath,amsfonts,amssymb,amsthm,mathtools}
\usepackage{wrapfig}


\usepackage{hyperref}
\usepackage[rgb]{xcolor}
\hypersetup{				% Гиперссылки
    colorlinks=true,       	% false: ссылки в рамках
	urlcolor=blue          % на URL
}
\author{Федоров Владимир}
\title{Общие принципы и математика в \LaTeX{}}
\date{\today}

\begin{document}

\maketitle

{ % в таких скобках можно локально поменять цвет ссылок
  \hypersetup{linkcolor=blue}
  \tableofcontents
}

\newpage %переводит на другую страницу
\section{Работа с текстом}
Привет, мир!

Новый абзац\\
Новая \hspace{1cm} строчка

Важное можно выделить \textbf{жирным} или \textit{курсивом},
даже \underline{подчеркнуть}\\
Слово в \fbox{рамочке}\\
Кавычки -- это <<вот>>\\
{\scriptsize Маленький шрифт} {\Large Большой шрифт}

\begin{center}
Этот текст выровнен по центру
\end{center}

\begin{flushright}
Этот текст выровнен по правому углу.
\end{flushright}

Сейчас будет список

\begin{enumerate} %список с нумерацией
\item Это
\item Список
\item с нумерацией
\begin{itemize} %список с кружочками
\item а это
\item список
\item с кружочками
\end{itemize}
\end{enumerate}

\section{Математика}

\subsection{Мир формул} %новый раздел



Моя первая формула $2+2=4$, ok

\[2+2=4\] %формула в середине

\begin{equation}\label{pifagor} 
a^2+b^2=c^2   %формула с нумерацией
\end{equation}

А вот номер формулы \eqref{pifagor}

\subsection{Дроби} %подраздел без нумерации

$\frac{1}{3} + \frac{1}{3} = \frac{2}{3}$ и тут чёт с высотой строки не так\\
Так красивее\footnote{но это неточно}
\[\frac{1}{3} + \frac{1}{3} = \frac{2}{3}\] 

\subsection{Скобки}

\[\left(\frac{4}{3} + 3\right)\cdot 5=\frac{65}{3}\] %тут нюанс с высотой скобок

\subsection{Индексы}
\[m_1, m_{12}, c^{22}\]

\subsection{Стандартные функции}
\[\sin x=0\]
\[\arctg x=\sqrt{3}\]
\[\log_{x-2}{(x^2-3x+4)} \geqslant 2\]

\subsection{Функции покрупнее}
\[\sum_{i=1}^{n}a_i+b_i\]
\[I = \int r^2dm\]
\[I = \int\limits_{0}^{1} r^2dm\]

\subsection{Диакритические знаки}

\[\dot{x} = 0,\]
\[\tilde{a}=\overline{bcde}\]
\[\overrightarrow{a}(0,3,4)\]
\[\underbrace{1+2+3+\dots+n}_{n}=N\]
\[(x-1)(x+1)>0 \stackrel{x>0}\Longleftrightarrow x-1>0\]

\subsection{Буквы алфавитов и математические шрифты}

\[\omega = \frac{2\pi}{T}\]
\[\overrightarrow{a} = \mathbf{a}\]
\[x\in \mathbb{R},\]
\[m_{\text{груза}}=15~\text{кг}\]
А вот кстати матрица, о ней скоро много узнаешь
\[\begin{pmatrix}
a_{11} & a_{12} & a_{13}\\
a_{21} & a_{22} & a_{23}\\
a_{31} & a_{32} & a_{33}
\end{pmatrix}\]

\subsection{Группировка формул}

\begin{equation}
\left\{
\begin{aligned}
&2\times a=4,\\
&-3\times b=6,\\
&-100\times c=110.
\end{aligned} \right.
\end{equation}

\[
\left.
\begin{aligned}
2\times a=4&,\\
-3\times b=6&,\\
-100\times c=110&.
\end{aligned} \right\} \Rightarrow -6ab = 24
\]

\section{Таблицы}

\begin{tabular}{|c|c|c|p{5cm}|}
\hline 
\multicolumn{3}{|c|}{Погрешности} & Подробнейший комментарий нашего Таёжного брата \\ 
\hline 
Систематическая & Случайная & Итог &\\ 
\hline 
0,03 & 0,04 & 0,05 & \\ 
\hline 
\end{tabular} 

\section{Обтекание}

\begin{wraptable}{r}{5cm}
\begin{tabular}{|c|c|c|}
\hline 
A & B & C \\ 
\hline 
1 & 2 & 3 \\ 
\hline 
\end{tabular} 
\caption{Бесполезная таблица}
\end{wraptable}
Lorem ipsum dolor sit amet, consectetur adipiscing elit, sed do eiusmod tempor incididunt ut labore et dolore magna aliqua. Ut enim ad minim veniam, quis nostrud exercitation ullamco laboris nisi ut aliquip ex ea commodo consequat. Duis aute irure dolor in reprehenderit in voluptate velit esse cillum dolore eu fugiat nulla pariatur. Excepteur sint occaecat cupidatat non proident, sunt in culpa qui officia deserunt mollit anim id est laborum


\section{Ссылки}

\begin{itemize}
\item С.М. Львовский \LaTeX : подробное описание;
\item Топовый курс от ВШЭ \href{https://www.coursera.org/learn/latex}{вот он}
\end{itemize}

\end{document}